\documentclass[a4paper]{article}
\usepackage{graphics,eurosym,latexsym}
\usepackage{listings}
\lstset{columns=fixed,basicstyle=\ttfamily,numbers=left,numberstyle=\tiny,stepnumber=5,breaklines=true}
\usepackage{times}
\usepackage[round]{natbib}
\bibliographystyle{plainnat}
\oddsidemargin=0cm
\evensidemargin=0cm
\newcommand{\be}{\begin{enumerate}}
\newcommand{\ee}{\end{enumerate}}
\newcommand{\bi}{\begin{itemize}}
\newcommand{\ei}{\end{itemize}}
\newcommand{\I}{\item}
\newcommand{\ty}{\texttt}
\textwidth=16cm
\textheight=23cm
\begin{document}
\title{\ty{Epos} \input{version}: Estimating Population Sizes and Allele Ages from
  Site Frequency Spectra}
\author{Bernhard Haubold\\\small Max-Planck-Institute for Evolutionary Biology, Pl\"on, Germany}
\maketitle
\section{Introduction} 
\ty{Epos} estimates historical population sizes based on site
frequency spectra. A site frequency spectrum is computed from a
haplotype sample. Table~\ref{tab:hap} shows a sample of $n=4$ haplotypes,
$h_1$--$h_4$, with $S=8$ segregating (polymorphic)
sites, $s_1$--$s_8$. Each segregating site consists of a column of four zeros and ones, where the
zeros indicate the ancestral state and the ones a mutation. We can
count the number of sites where one, two, or three haplotypes are mutated. This is called
the site frequency spectrum (SFS) of the sample, and Table~\ref{tab:sfs}A shows the spectrum
for our example data. There are seven mutations affecting a
single haplotype (singletons), zero mutations affecting two haplotypes
(doubletons), and one mutation affecting three haplotypes (tripleton).
\begin{table}
  \caption{Four example haplotypes}\label{tab:hap}
  \begin{center}
    \begin{tabular}{lcccccccc}\hline
haplotype   &   $s_1$ & $s_2$ & $s_3$ & $s_4$ & $s_5$ & $s_6$ & $s_7$ & $s_8$\\\hline
$h_1$ &      \ty{1} & \ty{0} & \ty{0} & \ty{0} & \ty{0} & \ty{0} & \ty{0} & \ty{0}\\
$h_2$ &      \ty{0} & \ty{1} & \ty{0} & \ty{0} & \ty{1} & \ty{0} & \ty{0} & \ty{1}\\
$h_3$ &      \ty{0} & \ty{1} & \ty{0} & \ty{1} & \ty{0} & \ty{1} & \ty{0} & \ty{0}\\
$h_4$ &      \ty{0} & \ty{1} & \ty{1} & \ty{0} & \ty{0} & \ty{0} & \ty{1} &
      \ty{0}\\\hline
      \end{tabular}
  \end{center}
\end{table}
In many empirical data sets it is not possible to distinguish between
segregating sites with $r$ mutations and those with $n-r$
mutations. In this case the spectrum is \textit{folded} by adding the
number of sites affecting $r$ haplotypes to the number of sites
affecting $n-$ haplotypes. Table~\ref{tab:hap}B shows the folded
version of the spectrum in Table~\ref{tab:hap}A: The single tripleton
is added to the seven singletons, while the number of doubletons
remains unchanged.
In addition to population sizes, \ty{epos} can also compute the
average ages of mutations affecting $1, 2,...,n-1$ haplotypes.

\begin{table}
  \caption{Folded (\textbf{A}) and unfolded (\textbf{B}) site
    frequency spectrum corresponding to the haplotype sample shown in Table~\ref{tab:hap}}\label{tab:sfs}
  \begin{center}
  \begin{tabular}{cc}
    \textbf{A} & \textbf{B}\\
    \begin{tabular}{cc}
      \hline
      $r$ & $f(r)$\\\hline
      1 & 7\\
      2 & 0\\
      3 & 1\\\hline
    \end{tabular}
    &
    \begin{tabular}{cc}
      \hline
      $r$ & $f(r)$\\\hline
      1 & 8\\
      2 & 0\\
      \hline
    \end{tabular}
  \end{tabular}
  \end{center}
\end{table}

\ty{Epos} implements theory developed by Michael Lynch (Arizona
State University) and Peter Pfaffelhuber (Freiburg University). The
theory is similar to the method developed by \cite{liu15:exp} and will
be described in a forthcoming paper.

\section{Getting Started}
\ty{Epos} was written in C on a computer running Linux. It depends on
two libraries, the Gnu Scientific Library
(\ty{lgsl}), and the Basic Linear Algebra Subprograms (\ty{lblas}). 
Please contact \ty{haubold@evolbio.mpg.de} if there are any problems with the program.
\bi
\I Change into the \ty{epos} directory
\begin{verbatim}
cd epos
\end{verbatim}
and list its contents
\begin{verbatim}
ls
\end{verbatim}
\I Generate \ty{epos}
\begin{verbatim}
make
\end{verbatim}
\I List its options
\begin{verbatim}
./epos -h
\end{verbatim}
\I Run tests
\begin{verbatim}
bash scripts/test.sh
\end{verbatim}
\ei

\section{Tutorial}
I first explain how to test \ty{epos} using simulated
data, and then how it can be applied to real data.
\subsection{Simulated Data}
\ty{Epos} was developed for estimating variable population
sizes. Nevertheless, we begin by simulating simple constant-size
scenarios before generating samples under models with varying population
sizes.
\subsubsection{Constant Population Size}
\begin{itemize}
\item We simulate samples with
  $n=30$ haplotypes using the well-known coalescent simulator \ty{ms} \citep{hud02:gen}:
\begin{verbatim}
ms 30 1 -t 10 
\end{verbatim}
These can automatically be converted to site frequency spectra using
my program \ty{sfs}, which is available form the same 
\ty{github} page as (\ty{evolbioinf}):
\begin{verbatim}
ms 30 1 -t 10 | sfs
\end{verbatim}
This spectrum is used by \ty{epos} to estimate population
sizes:
\begin{verbatim}
ms 30 1 -t 10 | sfs | epos -U -l 1000
#InputFile:	stdin
#Polymorphic sites surveyed:	48
#Monomorphic sites surveyed:	952
#LogLik:		5581.73
#Level  T[Level]  N[T]
2       2.34e+06  6.06e+05
\end{verbatim}
where \ty{-U} an unfolded site frequency spectrum, and \ty{-l} is the
sequence length, $l=1000$. \ty{Epos} interprets the population
mutation parameter, $\theta=4N_{\rm e}\mu$, as per-site, which means that under constant population size
the expected effective size is
\[
E[N_{\rm e}]=\frac{\theta}{4\mu l}.
\]
Since \ty{epos} uses by default $\mu=5\times 10^{-9}$, 
the expected population size for our simulation data
 is therefore 500,000. We can check this
by simulating 1000 samples, calculating a site frequency
spectrum for each, and averaging over the estimated population sizes:
\begin{verbatim}
ms 30 1000 -t 10 |              # generate 1000 samples of 30 haplotypes
sfs -i           |              # compute spectra for individual samples
epos -U -l 1000  |              # estimate population sizes
grep -v '^#'     |              # remove hashed lines
awk '{s+=$3;c++}END{print s/c}' # compute average population size
502000
\end{verbatim}
which is close to the expected 500,000. Let's see if folding the
spectrum changes this result by adjusting the options for \ty{sfs} and
\ty{epos}:
\begin{verbatim}
ms 30 1000 -t 10 | 
sfs -f -i        |  # fold the spectra of individual samples
epos -l 1000     |  
grep -v '^#'     |
awk '{s+=$3;c++}END{print s/c}'
481000
\end{verbatim}
This is again very similar to the expected value of 500,000.
\end{itemize}
\subsubsection{Variable Population Size}
\begin{itemize}
  \item For estimating variable population sizes we need the program \ty{epos2plot} to
    plot size as a function of time. \ty{Epos2plot} is also available form the \ty{evolbioinf} page
    on \ty{github}. We begin again by simulating samples under constant population
    size, but this time add plotting
\begin{verbatim}
ms 30 1000 -t 10 | 
sfs -i           | 
epos -U -l 1000  | 
epos2plot > epos1.dat # generate data ready for plotting
\end{verbatim}
Draw the graph by applying the program
\ty{gnuplot}\footnote{\ty{http://www.gnuplot.info/}} to the file
\ty{fig1.gp}, which is part of the \ty{epos} package:
\begin{verbatim}
gnuplot -p fig1.gp
\end{verbatim}
to get Figure~\ref{fig:con}. The fit between the estimated median size
and the true size is excellent for most of the plot, though it drifts
upward in the very distant past. It is important to realize that the number of samples
from which the estimates are derived declines into the past. Near the
present we have
\begin{verbatim}
head -n 5 epos1.dat 
#Time  LowerQ  Median  UpperQ    SampleSize
0      25300   484000  1.81e+07  1000
128    25300   484000  1.81e+07	 1000
324    26700   489000  1.81e+07	 1000
345    29800   492000  1.81e+07	 1000
\end{verbatim}        
while the furthest point in the past is usually computed from a single
sample:
\begin{verbatim}
tail -n 5 epos1.dat 
6.58e+06  2.34e+06  2.36e+06  2.71e+06  5
6.6e+06   2.34e+06  2.4e+06   2.71e+06  4
7.06e+06  2.34e+06  2.4e+06   2.71e+06  3
7.21e+06  2.4e+06   2.71e+06  2.71e+06  2
7.3e+06   2.71e+06  2.71e+06  2.71e+06  1
\end{verbatim}
Also, the variation in Figure~\ref{fig:con} is
large, especially close to the present.
\begin{figure}
  \begin{center}
    \scalebox{0.6}{\input{con}}
  \end{center}
  \caption{Estimating constant population size. For details see
  text.}\label{fig:con}
\end{figure}
\item Next, we simulate the scenario in Figure 2b of \cite{liu15:exp}
  with a single, approximately threefold change in population size
  from 7778 to 25636, which takes place 6809 generations in the
  past. Since this simulation includes substantial recombination, we
  replace \ty{ms} by its fast re-implementation, \ty{mspms}
  \citep{kel16:eff}:
\begin{verbatim}
mspms 30 1000 -t 12310 -r 9750 10000000 -eN 0.066 0.3 |
sfs -i                                                |
epos -u 1.2e-8 -l 10000000 -U                         |
epos2plot > epos2.dat
\end{verbatim}
This takes about 11 minutes. Notice the \ty{-u} option for \ty{epos},
which specifies the mutation rate per generation used by \cite{liu15:exp}, $\mu=1.2\times
10^{-8}$. Plot the result
\begin{verbatim}
gnuplot -p fig2.gp
\end{verbatim}
to get Figure~\ref{fig:2b}. The fit between the median estimated
population size and its true value remains excellent. However, the
variation in estimates is again large, particularly toward the present.
\begin{figure}
  \begin{center}
    \scalebox{0.6}{% GNUPLOT: LaTeX picture with Postscript
\begingroup
  \makeatletter
  \providecommand\color[2][]{%
    \GenericError{(gnuplot) \space\space\space\@spaces}{%
      Package color not loaded in conjunction with
      terminal option `colourtext'%
    }{See the gnuplot documentation for explanation.%
    }{Either use 'blacktext' in gnuplot or load the package
      color.sty in LaTeX.}%
    \renewcommand\color[2][]{}%
  }%
  \providecommand\includegraphics[2][]{%
    \GenericError{(gnuplot) \space\space\space\@spaces}{%
      Package graphicx or graphics not loaded%
    }{See the gnuplot documentation for explanation.%
    }{The gnuplot epslatex terminal needs graphicx.sty or graphics.sty.}%
    \renewcommand\includegraphics[2][]{}%
  }%
  \providecommand\rotatebox[2]{#2}%
  \@ifundefined{ifGPcolor}{%
    \newif\ifGPcolor
    \GPcolortrue
  }{}%
  \@ifundefined{ifGPblacktext}{%
    \newif\ifGPblacktext
    \GPblacktexttrue
  }{}%
  % define a \g@addto@macro without @ in the name:
  \let\gplgaddtomacro\g@addto@macro
  % define empty templates for all commands taking text:
  \gdef\gplbacktext{}%
  \gdef\gplfronttext{}%
  \makeatother
  \ifGPblacktext
    % no textcolor at all
    \def\colorrgb#1{}%
    \def\colorgray#1{}%
  \else
    % gray or color?
    \ifGPcolor
      \def\colorrgb#1{\color[rgb]{#1}}%
      \def\colorgray#1{\color[gray]{#1}}%
      \expandafter\def\csname LTw\endcsname{\color{white}}%
      \expandafter\def\csname LTb\endcsname{\color{black}}%
      \expandafter\def\csname LTa\endcsname{\color{black}}%
      \expandafter\def\csname LT0\endcsname{\color[rgb]{1,0,0}}%
      \expandafter\def\csname LT1\endcsname{\color[rgb]{0,1,0}}%
      \expandafter\def\csname LT2\endcsname{\color[rgb]{0,0,1}}%
      \expandafter\def\csname LT3\endcsname{\color[rgb]{1,0,1}}%
      \expandafter\def\csname LT4\endcsname{\color[rgb]{0,1,1}}%
      \expandafter\def\csname LT5\endcsname{\color[rgb]{1,1,0}}%
      \expandafter\def\csname LT6\endcsname{\color[rgb]{0,0,0}}%
      \expandafter\def\csname LT7\endcsname{\color[rgb]{1,0.3,0}}%
      \expandafter\def\csname LT8\endcsname{\color[rgb]{0.5,0.5,0.5}}%
    \else
      % gray
      \def\colorrgb#1{\color{black}}%
      \def\colorgray#1{\color[gray]{#1}}%
      \expandafter\def\csname LTw\endcsname{\color{white}}%
      \expandafter\def\csname LTb\endcsname{\color{black}}%
      \expandafter\def\csname LTa\endcsname{\color{black}}%
      \expandafter\def\csname LT0\endcsname{\color{black}}%
      \expandafter\def\csname LT1\endcsname{\color{black}}%
      \expandafter\def\csname LT2\endcsname{\color{black}}%
      \expandafter\def\csname LT3\endcsname{\color{black}}%
      \expandafter\def\csname LT4\endcsname{\color{black}}%
      \expandafter\def\csname LT5\endcsname{\color{black}}%
      \expandafter\def\csname LT6\endcsname{\color{black}}%
      \expandafter\def\csname LT7\endcsname{\color{black}}%
      \expandafter\def\csname LT8\endcsname{\color{black}}%
    \fi
  \fi
    \setlength{\unitlength}{0.0500bp}%
    \ifx\gptboxheight\undefined%
      \newlength{\gptboxheight}%
      \newlength{\gptboxwidth}%
      \newsavebox{\gptboxtext}%
    \fi%
    \setlength{\fboxrule}{0.5pt}%
    \setlength{\fboxsep}{1pt}%
\begin{picture}(7200.00,6720.00)%
    \gplgaddtomacro\gplbacktext{%
      \csname LTb\endcsname%%
      \put(1078,704){\makebox(0,0)[r]{\strut{}\large $10^{3}$}}%
      \put(1078,2636){\makebox(0,0)[r]{\strut{}\large $10^{4}$}}%
      \put(1078,4567){\makebox(0,0)[r]{\strut{}\large $10^{5}$}}%
      \put(1078,6499){\makebox(0,0)[r]{\strut{}\large $10^{6}$}}%
      \put(1210,484){\makebox(0,0){\strut{}\large $10^{2}$}}%
      \put(3282,484){\makebox(0,0){\strut{}\large $10^{3}$}}%
      \put(5355,484){\makebox(0,0){\strut{}\large $10^{4}$}}%
    }%
    \gplgaddtomacro\gplfronttext{%
      \csname LTb\endcsname%%
      \put(198,3601){\rotatebox{-270}{\makebox(0,0){\strut{}\large $N_{\rm e}$}}}%
      \put(4006,154){\makebox(0,0){\strut{}\large Time (Generations)}}%
      \csname LTb\endcsname%%
      \put(5816,6326){\makebox(0,0)[r]{\strut{}\large 5\% quantile}}%
      \csname LTb\endcsname%%
      \put(5816,6106){\makebox(0,0)[r]{\strut{}\large median}}%
      \csname LTb\endcsname%%
      \put(5816,5886){\makebox(0,0)[r]{\strut{}\large expected}}%
      \csname LTb\endcsname%%
      \put(5816,5666){\makebox(0,0)[r]{\strut{}\large 95\% quantile}}%
    }%
    \gplbacktext
    \put(0,0){\includegraphics{fig2b}}%
    \gplfronttext
  \end{picture}%
\endgroup
}
  \end{center}
  \caption{Estimating population sizes under a model with one
    instantaneous change. See text for details.}\label{fig:2b}
\end{figure}
\item As a last example, we simulate haplotypes under
  the exponential growth scenario \cite{liu15:exp} used in their
  Figure 2e:
    \small
\begin{verbatim}
mspms 30 1000 -t 432000 -r 340000 10000000 -G 46368 -eN 0.0001027 0.008889 |
sfs -i                                                                     |
epos -u 1.2e-8 -l 10000000 -U                                              |
epos2plot > epos3.dat
\end{verbatim}
\normalsize
Plot this
\begin{verbatim}
gnuplot -p fig3.gp
\end{verbatim}
to get Figure~\ref{fig:2e}, where the estimation is less
accurate than in the one-step scenario in Figure~\ref{fig:2b}.
This illustrates the difficulty of accurately capturing population size
changes in the recent past.
\begin{figure}
  \begin{center}
    \scalebox{0.6}{\input{fig2e}}
  \end{center}
  \caption{Estimating population sizes under an exponential growth
    model. See text for details.}\label{fig:2e}
\end{figure}
\end{itemize}

\subsection{Real Data}
\begin{itemize}
\item As an example for empirical data we use a site frequency spectrum obtained from
  the Kap population of \textit{Daphnia pulex}:
\begin{verbatim}
head kap144i.dat 
0	185297
1	1987
2	1138
3	851
4	729
5	672
6	542
7	509
8	459
9	430
\end{verbatim}
Notice the ``zero-class'', which does not appear in the
example spectra in Tables~\ref{tab:sfs}A and B. The zero-class gives
the number of monomorphic sites. If a spectrum contains a zero-class,
the sequence length is simply the sum of all counts, and there is no need
to pass the sequence length to \ty{epos} using \ty{-l}. Moreover, the
site frequency spectrum of Kap is folded, i. e. there is no need for \ty{-U}. Hence
we analyze our example data without any options:
\begin{verbatim}
epos kap144i.dat
#InputFile:	../data/kap144i.dat
#Polymorphic sites surveyed:	18037
#Monomorphic sites surveyed:	185297
#LogLik:		2.1501e+06
#Level  T[Level]  N[T]
25      5.18e+04  3.73e+05
2       3.90e+06  1.00e+06
\end{verbatim}
\item This only gives a single point estimate, and in the absence of
  further samples it is hard to judge its reliability. However,
  \ty{epos} implements the bootstrap to investigate the robustness of
  results based on single samples. To run 1000 bootstrap replicates, enter
\begin{verbatim}
epos -b 10000 kap144i.dat | epos2plot > epos4.dat
\end{verbatim}
This takes about five minutes. Plot the result
\begin{verbatim}
gnuplot -p fig4.gp
\end{verbatim}
to get Figure~\ref{fig:kap}.
\begin{figure}
  \begin{center}
    \scalebox{0.6}{\input{kap}}
  \end{center}
  \caption{Estimating the population size of the \textit{D. pulex} Kap
    population.}\label{fig:kap}
\end{figure}
\end{itemize}

\section{The Averge Age of an Allele}
\begin{itemize}
  \item The expected
    time to the most recent common
    ancestor in a constant size population is
    \[
    E[T_{\rm MRCA}]=2\left(1-\frac{1}{n}\right),
    \]
    where time is measured in $2N$
    generations  \citep[p.76]{wak09:coa}. Now, if $n=2$, $E[T_{\rm
        MRCA}]=1$. We simulate a site frequency spectrum with $n=2$,
\begin{verbatim}
ms 2 1 -t 10 | sfs > test.sfs
\end{verbatim}
estimate the population size,
\begin{verbatim}
epos -U -l 1000 test.sfs
#InputFile:	test.sfs
#Polymorphic sites surveyed:	15
#Monomorphic sites surveyed:	985
#LogLik:		5804.25
#Level  T[Level]  N[T]
2       1.50e+06  7.50e+05
\end{verbatim}
and estimate the age of singleton alleles
\begin{verbatim}
epos -U -l 1000 -a test.sfs
#InputFile:	test.sfs
#Polymorphic sites surveyed:	15
#Monomorphic sites surveyed:	985
#r  A[r]
1   1.5e+06
\end{verbatim}
which is twice the population size, as expected, given that
\ty{epos} estimates time in units of generations.
\item For larger samples the population sizes might look like this
\begin{verbatim}
ms 30 1 -t 1000 | sfs | tee test.sfs | epos -U -l 10000000 
#InputFile:	stdin
#Polymorphic sites surveyed:	4169
#Monomorphic sites surveyed:	9.99583e+06
#LogLik:		1.51132e+08
#Level  T[Level]  N[T]
16      1.30e+01  9.73e+01
8       3.62e+03  1.18e+04
2       1.94e+04  4.59e+03
\end{verbatim}
and the corresponding allele sizes like that
\begin{verbatim}
epos -U -l 10000000 -a test.sfs 
#InputFile:	test.sfs
#Polymorphic sites surveyed:	4169
#Monomorphic sites surveyed:	9.99583e+06
#r  A[r]
1   3117.22
2   3741.8
3   4455.55
...
29  19359.7
\end{verbatim}
\item Finally, we can estimate the average age of alleles for the Kap
  population:
\begin{verbatim}
epos -a .kap144i.dat | head
#InputFile:	../data/kap144i.dat
#Polymorphic sites surveyed:	18037
#Monomorphic sites surveyed:	185297
#r  A[r]
1   190499
2   309342
3   396787
...
72  2.68069e+06
\end{verbatim}  
\end{itemize}

\section{Change Log}
\begin{itemize}
\item Version 0.1 (Oct. 25, 2017)
  \begin{itemize}
  \item First running version based on the GSL.
  \end{itemize}
\item Version 0.2 (Oct 26, 2017)
  \begin{itemize}
    \item Used \ty{LAPACKE\_dgesv} in \ty{getPopSizes}; makes no
      difference compared to the previous version.
  \end{itemize}
\item Version 0.3 (Oct 26, 2017)
  \begin{itemize}
    \item Used \ty{LAPACKE\_dgesvx} in \ty{getPopSizes}; makes no
      difference compared to previous version.
  \end{itemize}
\item Version 0.4 (Oct 27, 2017)
  \begin{itemize}
  \item Allow construction of SFS based on \ty{-t} switch.
  \item Allow switching between the GSL algorithm (\ty{-g}) and the
    default LAPACK algorithm. 
  \end{itemize}
\item Version 0.5 (November 2, 2017)
  \begin{itemize}
    \item Allow the straight usage of the trapezoid matrix for solving
      the system (\ty{-T}).
    \item Print out coefficient matrix (\ty{-p}).
  \end{itemize}
\item Version 0.6 (November 10, 2017)
  \begin{itemize}
  \item Use \ty{long double} in \ty{getPopSizesTri2}; did not help.
  \end{itemize}
\item Version 0.7 (November 10, 2017)
  \begin{itemize}
    \item Implement Peter Pfaffelhuber's formula; working only partially.
  \end{itemize}
\item Version 0.8 (November 10, 2017)
  \begin{itemize}
    \item Peter's equation (6) is working.
  \end{itemize}
\item Version 0.9 (November 10, 2017)
  \begin{itemize}
  \item Peter's equation in arbitrary precision in MPFR
    library. Numerical stability achieved with 329 bits per number.
  \end{itemize}
\item Version 0.10 (November 30, 2017)
  \begin{itemize}
  \item Implemented equation (3) from Peter's memo dated Nov. 13. Not
    working.
  \end{itemize}
\item Version 0.11 (December 1, 2017)
  \begin{itemize}
  \item Implemented revised equation (3) from Peter's memo dated Nov. 30. Not
    working.
  \end{itemize}
\item Version 0.12 (December 15, 2017)
  \begin{itemize}
    \item Implemented Estimator 2.1 from Peter's memo dated Dec. 2,
      2017. Code is running, but the results look odd.
  \end{itemize}
\item Version 0.13 (December 16, 2017)
  \begin{itemize}
  \item Fixed the implementation of Estimator 2.1; results OK now.
  \end{itemize}
\item Version 0.14 (December 18, 2017)
  \begin{itemize}
  \item Implemented optimization strategy for folded SFS; working.
  \item Implemented optimization strategy for unfolded/even SFS;
    working.
  \item Implemented optimization strategy for folded/odd SFS; not
    working yet.
  \end{itemize}
\item Version 0.15 (December 18, 2017)
  \begin{itemize}
  \item Fixed error in search for optimal number of steps.
  \end{itemize}
\item Version 0.16 (December 19, 2017)
  \begin{itemize}
  \item Fixed error in left-hand side of folded/odd equation; working.
  \item Changed search strategy.
  \item Changed default value of \ty{-d} from $10^{-6}$ to $10^{-3}$.
  \item Simplified user interface.
  \item Included set of test cases (\ty{test.sh}).
  \end{itemize}
  \item Version 0.17 (December 20, 2017)
  \begin{itemize}
  \item Fixed searching routine.
  \item Changed default value of \ty{-d} from $10^{-3}$ to $10^{-2}$.
  \item Included addition of the $\lambda$-factor; seems to make no difference.
  \end{itemize}
\item Version 0.18 (December 20, 2017)
  \begin{itemize}
  \item Fixed the $\lambda$-factor; computation is now much
    stabilized.
  \end{itemize}
\item Version 0.19 (January 11, 2018)
  \begin{itemize}
  \item Included the $\lambda$-factor in the computation of
    $\Psi$. Computations now applicable to real data.
  \item Changed the default-value of $\lambda$ from $10^{-7}$ to $10^{-5}$.
  \end{itemize}
\item Version 0.20 (January 11, 2018)
  \begin{itemize}
  \item Consider zero-class mutations in computation, if present.
  \item Changed default $\lambda$ to $2\times 10^{-5}$ to get all data
    sets to run.
  \end{itemize}
\item Version 0.21 (January 13, 2018)
  \begin{itemize}
    \item Reverted output of levels, going from the present into the
      past.
    \item Default output is now as a function of times instead as a
      function of levels.
    \item Included ``step-wise'' option for plotting times and levels.
    \item Fixed time computation.
    \item Included error message for negative population sizes.
    \item Removed memory leaks and other subtle bugs using \ty{valgrind}.
  \end{itemize}
\item Version 0.22 (January 16, 2018)
  \begin{itemize}
    \item Fixed important bug in function foldedEpsi, where variable
      \ty{b} was computed as a function of \verb+sfs->f[n/2]+ instead
      of, now \verb+sfs->f[n/2-1]+.
  \end{itemize}
\item Version 0.23 (January 16, 2018)
  \begin{itemize}
  \item Search for optimal $\lambda$.
  \item Allow arbitrary level as first entry in level list.
  \item Output $\lambda$, $\Psi$, and the levels added to make it
    easier to follow the program.
  \item Reduced program to folded/even case.
  \end{itemize}
\item Version 0.23 (January 17, 2018)
  \begin{itemize}
  \item Catch GSL-exceptions.
  \item Changed output format
  \end{itemize}
\item Version 0.24 (January 18, 2018)
  \begin{itemize}
  \item Not quite sure what changed.
  \end{itemize}
\item Version 0.25 (January 18, 2018)
  \begin{itemize}
  \item Fixed bug in computation of the $\lambda$-term in
    \ty{foldedEpsi}.
  \end{itemize}
\item Version 0.26 (January 19, 2018)
  \begin{itemize}
  \item Set $\lambda=0$ and add levels until negative population sizes
    appear. This is fast and appears to be effective.
  \end{itemize}
\item Version 0.27 (January 24, 2018)
  \begin{itemize}
  \item Output number of polymorphic and monomorphic sites surveyed.
  \end{itemize}
\item Version 0.28 (???)
\item Version 0.29 (January 28, 2018)
  \begin{itemize}
  \item Fixed missing resetting of times during iteration over files.
  \item Removed superfluous option for step-wise output (\ty{-s}).
  \item Output name of input file.
  \item Removed inclusion of \ty{mpfr.h} from \ty{epos.c}.
  \item Removed search for the initial level to add; by definition
    this must be 2, i. e. one population size for the entire coalescent.
  \end{itemize}
\item Version 0.30 (January 31, 2018)
  \begin{itemize}
  \item Fixed computation of the mutation rate. Previously I
    multiplied the per site mutation rate with the number of
    monomorphic positions. Now it is mutated by the number of all
    positions.
  \item Added bootstrapping.
  \end{itemize}
\item Version 0.31 (February 1, 2018)
  \begin{itemize}
    \item Computation of mutation rate was correct in previous
      version, after all, so reverted to that.
  \end{itemize}
\item Version 0.32 (February 7, 2018)
  \begin{itemize}
    \item Introduced the \ty{-m} switch.
  \end{itemize}
\item Version 0.33 (February 7, 2018)
  \begin{itemize}
    \item Fixed $\delta$ computation in function \ty{delta} in
      \ty{util.c}. This reduces the computation for $m=1$ to
      Watterson's estimator, as expected.
  \end{itemize}
\item Version 0.34 (February 9, 2018)
  \begin{itemize}
    \item Reintroduced unfolded spectrum (\ty{-u}) and compared to the equations
      in Peter's memo of December 19, 2017.
    \item Reintroduced $\lambda$ and set it by default to $10^{-7}$.
    \item Reintroduced $\delta$ and set it by default to $0.0$.
    \item Fixed sample size computation at the end of \ty{sfs.c}.
  \end{itemize}
\item Version 0.35 (February 9, 2018)
  \begin{itemize}
  \item Introduced estimation of $\lambda=1./\mu/\texttt{args->f}$.
  \end{itemize}
\item Version 0.36 (February 14, 2018)
  \begin{itemize}
    \item Changed setting of $\lambda$ to
      $\lambda=\mu\times\texttt{args->f}$. By default
      $\texttt{args->f}=1$.
  \end{itemize}
\item Version 0.37 (February 16, 2018)
  \begin{itemize}
    \item Fixed sample size computation for folded/even in function
      \ty{getSfs} in \ty{sfs.c}.
  \end{itemize}
\item Version 0.38 (February 16, 2018)
  \begin{itemize}
  \item Included working \ty{-m} switch.
  \end{itemize}
\item Version 0.39 (February 19, 2018)
  \begin{itemize}
  \item Fixed passing of $\lambda$ in iterated runs.
  \item Fixed numerical underflow when multiplying with $\lambda$ in
    \ty{foldedEpsi} in \ty{foldedE.c}.
  \item Fixed numerical underflow when multiplying with $\lambda$ in
    \ty{psi} in \ty{unfolded.c}
  \item Included check for positive $\Psi$ in both cases.
  \item Expanded verbose output.
  \end{itemize}
\item Version 0.40 (February 21, 2018)
  \begin{itemize}
  \item Fixed error in \ty{foldedEpsi} in \ty{foldedE.c}
  \item Removed \ty{if(m > 1)} from \ty{getCoeffMat} in
    \ty{foldedE.c}.
  \item Switched \ty{n/2} in \ty{foldedEpsi} to \ty{n/2.}.
  \item Ensured that \ty{sfs->u} is always set to a value in
    \ty{getSfs}.
  \item Replace \verb+u = args->u+ by \verb+u = sfs->u+ in
    \ty{getCoeffMat} in \ty{unfolded.c}.
  \item Re-implemented \ty{foldedEpsi} in \ty{foldedE.c}
  \end{itemize}
\item Version 0.41 (February 22, 2018)
  \begin{itemize}
    \item Changed \ty{4.*u*(n/2.)} to \ty{4.*u/(n/2.)} in
      \ty{foldedEpsi} in \ty{foldedE.c}. This was a bug in the
      computation of $\Psi$ for the folded/even case.
  \end{itemize}
\item Version 0.42 (February 22, 2018)
  \begin{itemize}
    \item Removed line \ty{prevMinPsi = DBL\_MAX} in \ty{foldedE} in \ty{foldedE.c}.
  \end{itemize}
\item Version 0.43 (February 22, 2018)
  \begin{itemize}
    \item Added diagnostic output in case negative population sizes
      are found.
  \end{itemize}
\item Version 0.44 (February 22, 2018)
  \begin{itemize}
  \item $\Psi$ now also reported if only one level is included.
  \end{itemize}
\item Version 0.45 (February 23, 2018)
  \begin{itemize}
  \item Included the \ty{-n} option to allow negative population
    sizes.
  \end{itemize}
\item Version 0.46 (February 23, 2018)
  \begin{itemize}
    \item Fixed \ty{if(change > args->d)}-phrase in \ty{unfolded} and
      \ty{foldedE}. This lacked re-computation of the population sizes
      with the best new level added, and assignment of $\Psi$.
  \end{itemize}
\item Version 0.47 (February 26, 2018)
  \begin{itemize}
    \item Reorganized code to remove duplication. The searching for
      best population sizes is now done in only one place,
      \ty{getPopSizes} in \ty{popSizes.c}.
    \item Added printing of intermediate population sizes if the
      \ty{-V} option is used.
  \end{itemize}
\item Version 0.48 (March 2, 2018)
  \begin{itemize}
  \item Multi-threaded version.
  \end{itemize}
\item Version 0.49 (March 6, 2018)
  \begin{itemize}
  \item Reverted to single-threaded behavior by removing \ty{-t} from
    the options list and setting it to 1 in the background. This avoids the occasional
    race-conditions observed with the multi-threaded version.
  \end{itemize}
\item Version 0.50 (March 14, 2018)
  \begin{itemize}
  \item Find number of levels through cross-validation (\ty{-c}).
  \end{itemize}
\item Version 0.51 (March 14, 2018)
  \begin{itemize}
  \item Find lambda through cross-validation (\ty{-L}).
  \end{itemize}
\item Version 0.52 (March 17, 2018)
  \begin{itemize}
    \item Include reporting of negative population sizes in verbose
      output (\ty{-V}).
    \item Changed     \ty{prevMinPsi = prevMinPsi;} in
      \ty{getPopSizes} to \ty{prevMinPsi = currMinPsi;}.
  \end{itemize}
\item Version 0.53 (April 7, 2018)
  \begin{itemize}
  \item Always allow negative population sizes.
  \item Cross-validation by default.
  \item $\lambda=0$ by default.
  \item If negative population sizes are found, the program searches
    for optimal $\lambda$ by going through $\lambda=0..\mu$. This is
    slow and would need to be optimized in future versions.
  \item Added Scripts for extracting quantiles from \ty{epos} output.
  \end{itemize}
\item Version 0.54 (April 11, 2018)
  \begin{itemize}
    \item Fixed array out-of-bounds error in \ty{shuffleArr} in \ty{sfs.c}.
  \end{itemize}
\item Version 0.55 (April 12, 2018)
  \begin{itemize}
    \item ``Unfolded'' mode not working; so I removed that option for now.
  \end{itemize}
\item Version 0.56 (May 31, 2018)
  \begin{itemize}
  \item Fixed Error in documentation.
  \end{itemize}
\item June 13, 2018
  \begin{itemize}
    \item Posted \ty{epos} on \ty{github}. Please refer to the commit
      messages for details on subsequent changes.
  \end{itemize}
\end{itemize}
\bibliography{/home/haubold/References/references}
\end{document}

